From our analysis, we see that we can reject null hypothesis for total citation count, citations per year and $\Delta citations$. So, as we intuitively expected, there seems to be correlation between these measures of productivity and salary. Years since Ph.~D. is correlated with base salary (with R-squared value around $0.55$). This is to be expected, since professors must demonstrate sustained productivity for a period of time in order to qualify for promotions. From the results, we see that it adds approximately $2400$ USD per year since Ph.~D. Also, from the result in section~\secref{sectionDeltacit}, we see that recent research output is correlated with hikes in salary. Our empirical study suggests that each new citation translates to about $33$ USD for High Energy Physicists in UC. 

We found no correlation with PageRank. Eigenvector centrality is a sophisticated measure; universities probably do not factor pagerank as a measure for professors\textquoteright  research output. Also, since pagerank is normalized, as the number of papers increases over time, the pagerank of each paper decreases. Thus $\Delta(pagerank)$ is not useful. Instead of using PageRank, we could try to consider the importance of each citation by considering whether the paper was published in a top rank journal, mid-range journal or conference etc. But such a measure would be very subjective and hard to quantify, and so we could not include this in the perview of our study.

g and h-indices show no correlation with salary.  These indices are bounded by the number of papers. and both are too insensitive to recent publications (which would not affect the g-/h-index until they have accumulated many citations). We do not really expect to find a correlation between salary and the more sophisticated centrality measures. Since these measures are new, universities probably are not yet considering centrality when determining salaries. It is possible that these measures are considered while making decisions like hiring, promotions or giving awards. But, current data does not show any proof of them being used for determining salary.

It is also of our interest to compare salary attributed to centrality and seniority. According to the regression output, each additional year adds another $2400$ USD to the base salary, while a citation adds $5$ USD. At least for High Energy Physics, the effect of seniority overwhelms the effect of citations, since most professors in High Energy Physics do not get hundreds of citations per year.
