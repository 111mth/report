We assume a simple linear model with constant term.
\begin{equation}
salary = \beta_0 + \sum _{i=1}^{n} \beta_i x_i
\end{equation}
where $\beta_0$ is the intercept (constant) and $\beta_i$ (for $i \in \{1,2,\ldots,n\}$) are the coefficients associated with explanatory variables $x_i$. A key explanatory variable in each regression will be perceived \emph{research value} of a professor, which we shall refer to as primary test variable. There are other possible variables (\emph{secondary variables}, or \emph{controls}) which might influence salary, such as 
\begin{itemize}
\item Years since Ph.D.: Different ranks (assistant, associate, and full professor) have different base salaries. Typically it takes a few years to get promoted.
\item Area of research: Professors in management, finance, physics, biology, and the humanities have different typical salaries.
\item Prestige of the university: Scholars may value being at a highly ranked university, either for reasons of prestige, or for the access to better assistants and collaborators it provides. So universities with high reputations may not have to pay as much to attract good scholars. On the other hand, top universities may have greater resources available, and may sometimes be in a position to overpay.
\item Cost of living: Different areas have different amenities and costs of living, which affects the willingness of employees to accept a given salary.
\item Which state contains the university will also affect the salary, since each state has its own system of universities and hence differences in their funding and systems of payment.
\end{itemize}
