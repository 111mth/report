We assume a simple linear model.
\begin{equation}
salary = \beta_0 + \sum _{i=1}^{n} \beta_i x_i
\end{equation}

where $\beta_0$ is the intercept (constant) and $\beta_i$, $i = \{1,2,...n\}$ are the coefficients (slopes) associated with explanatory variables $x_i$. One of the explanatory variable will be perceived \emph{research value} of a professor, which we shall refer to as primary variable. There are other possible variables (\emph{secondary variables}) which might influence salary such as 
\begin{enumerate}
\item Years since Ph.D.: Different ranks, like assistant professor, associate professor, full professor have different base salary. And usually, it takes a few years to get promoted from one rank to another.
\item Area of research: Professors in Management and Finance might have different salary compared to professors in Physics or Biology or Humanities.  
\item Ranking of the University: Highly ranked universities themselves add value to the professor, may be in terms of the kind of graduate students he/she gets, co-workers, postdocs etc. So, such universities which high reputations do not require to pay too high to compete with other universities. Whereas second rung universities would want to attract good researches to join their university and hence would need to make better offers.
\item Living cost: Different areas have different living cost which is considered by the policy makers to decide the salary.
\item State in which the university is situated will also affect the salary since each state has its own system of universities and hence differences in their system of payment and so on.
\end{enumerate}
