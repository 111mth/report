It is almost impossible to retrieve a complete citation network of papers from every field. Due to the challenges of combining data from different domains and operating on a huge number of vertices and edges, we must restrict ourselves to a specific field of science.

Modern research is highly cross-domain, and citation network of any field is not self contained. There are always edges in and out of a particular field. We shall focus our work on the research area of Theoretical High Energy Physics since there is much more homogeneity in the citation behavior as compared to Networks? or Computational Biology.

We crawled the Theoretical High Energy Physics (Hep-TH) section of Arxiv.org, which is an e-print service that is highly popular for physicists. The dataset ranges from August 1991 to May 2012, with $86,165$ papers and $1,163,901$ citations strictly within the field.

Because private universities do not release salary data for individual professors, our set is restricted to professors in public universities. We confine ourselves to University of California system as it will minimize the variance of the cost of living. Further, we shall assume that the cost of living is same in California and thus drop the value from the list of secondary variables. We shall also ignore the ranking of the university in this work.

The salary database was obtained from University of California Data Analysis service, at ucpay.globl.org. The database consists of salary data for 10 schools in the UC system, from 2004 to 2010. After cross referencing the authors\textquoteright  names with professors employed by the UC system, we were able to identify 78 professors. By manual checking, we found that some of the professors are from other areas like Medicine, Political Science, and Mathematics. Even though they have published papers in High Energy Physics, these professors have bulk of other work apart from the field. Hence we cannot not compare them directly with professors of Physics.

The UC system releases two different types of salary: gross and base. On top of base pay, gross pay includes overtime, bonuses, housing and relocation allowances, administrative stipends, and others. This study only uses base pay, as it is relatively stable compared to the change in gross pay.

Finally, we do not consider professors with more than 40 years since Ph.D. in our study. Most of the aforementioned professors are emeritus, no longer receiving full time salary from the University. After removing emeritus professors and professors in different fields, our dataset consisted of fifty professors in total.
