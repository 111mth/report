\subsection{Research field}
It is infeasible to retrieve a complete citation network of all published research. In light of the challenges of gathering data from different domains and performing computations on a huge number of vertices and edges, we restricted the study to a specific field of science.

Research is highly interdisciplinary, so the citation network of any one field is not a disjoint connected component of the overall academic network. To minimize the number of relevant omitted edges, our study focuses on the research area of \emph{theoretical high-energy physics} (\textsc{hep-th}). Citations for this domain should be much more inner-directed than in highly interdisciplinary fields such as Networks or Computational Biology.

\subsection{Citation data}
We crawled the \textsc{hep-th} section of \texttt{arxiv.org}, an e-print service highly popular among physicists. Our dataset contains $86,165$ papers and $1,163,901$ citations strictly within the field, from August 1991 through May 2012.

\subsection{Salary data}
We are limited to professors in public universities which release salary data for individual professors. We confine ourselves to University of California (UC) system since this partially controls for an important secondary variable, location of the university, while it is still a very large system. Furthermore, this helps control for cost of living, as all schools are in California. We do not control for the prestige of the university apart from observing that all schools meet the standards of the UC system.

The salary database, obtained from University of California Data Analysis service at \url{http://ucpay.globl.org/}~\cite{ucpay}, contains salary data for the 10 schools in the UC system, from 2004 to 2010. After cross-referencing the names of authors of research papers with professors employed by the UC system, we found 78 professors. After removing emeritus professors and professors with primary fields other than physics\footnote{By manual checking, we found that several of the high-energy physics papers were by professors who worked mainly in other fields. These professors were eliminated because the exclusion of other fields would reduce the centrality we observe, thus biasing the regression results.}, we identified 50 professors.

\paragraph{Base vs. gross salary}
The University of California discloses gross and base salary for each professor. On top of base pay, gross pay includes overtime, bonuses, housing and relocation allowances, and additional compensation for summer research or administrative work. To the extent this additional compensation rewards high-quality research, it is part of what we seek to measure. However, the other components, as a source of variation in pay not based upon research output, potentially add noise to our tests. This would bias our regression estimates iff there is a correlation between research quality and the additional compensation. This study only uses base pay, as it is relatively stable compared to the change in gross pay.

\paragraph{Emeritus professors}
Finally, we do not consider professors with more than 40 years since Ph.D. in our study. Most of the aforementioned professors are emeritus, drawing only a pension despite many having high total research output. Their inclusion would bias the results.
