We studied the relations between perceived research output and academic salaries in Theoretical High Energy Physics, using the citation network from Arxiv.org and publicly available salary data from University of California system. We found evidence for correlation between years since Ph.D. and base salary. Recent research output seems to be correlated with hikes in salary.

We found no correlation with PageRank, g-index, and h-index under our model. This suggests that more complicated measures of centrality are not yet used for evaluating professors, and g-index, h-index are not a good measure for quantifying the change in productivity.

As a null hypothesis can never be proven, this study cannot prove that a correlation does not exist between complex measures of centrality and academic salary. To reduce the effect of outliers, we can incorporate additional areas of research and salary data in further studies. Also, we may need a different mathematical model.

It also might be of interest to examine the contribution of each professor in a paper. A paper in Experimental High Energy Physics might have thousands of citations, but also hundreds of authors. One of the reasons why we limited our study to Theoretical High Energy Physics is that it tended to have fewer co-authors. A study of co-authoring behavior across departments might be valuable for research universities.

We have not examined the value of citations coming from different authors. Intuitively, a citation from a renowned professor would be a better indication of one\textquoteright~s work, compared to a citation from an undergraduate student without any research experience. A model which assigns different weights to each citation, and also produces meaningful change over time is of interest.

Lastly, using our model, we can analyze the effect of different secondary variables. For instance, gender salary equity is an issue faced by most colleges and universities~\cite{becker1995}. Further analysis could include different areas of research, location of schools, and teaching evaluations.
