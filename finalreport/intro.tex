The problem of identifying how important a node is in a network is encountered in various applications. For example, search engines need to find the importance of web pages for a keyword and rank them in order to produce useful output for the user. For advertising and epidemic control, identifying the most important nodes in the network is a central piece of the obstacle as well.

Employers are also interested in evaluating the productivity of an employee, because people expect to be compensated based on the value of their work. In most environments, it is difficult to measure productivity, because the output of an individual is often combined with others\textquoteright  to yield the final product. However, in academia, citation networks provide the means to evaluate a professor\textquoteright s research quality. High quality papers are expected to have more citations, and indices like h-index, which intend to quantify scientific productivity based on publication record, are used in services like Google Scholar to augment the summary of a professor\textquoteright s research output.

In the present paper, given a citation network and salary data from ten public universities, we ask whether the real world behavior demonstrates the relationship between productivity and compensation. We evaluate different notions of centrality against the real world data, and study the relationship between the change in salary and the change in productivity measure. With the same model, we investigate how seniority and research performance affect the compensation. We hope this study would inspire the policy makers to re-evaluate their current measures of productivity, and the general audience in job markets to know how the system works in their field of interest.

With such widespread applications, people have asked these questions on academic salaries previously. Grofman~\cite{grofman2009political}and Gomez-Mejia et al.~\cite{gomez1992} have studied the determinants of faculty pay by correlating academic salaries with citation counts. They have found a correlation between citation counts and academic salary, but they mainly focus on total citation count and citations per year as a measure of research value.

However, citation counts are also imperfect, because it is better to be cited by an important author than by a minor author. Bollen, et al. argue that journal impact factors should incorporate centrality measurements like PageRank, which gives different weights for citations from papers with different importance. As suggested, we evaluate more sophisticated measures of centrality like PageRank, h-index, and g-index against the salary data.

The rest of the paper is organised as follows. First we introduce and motivate the general model for studying the correlation between different measures of centrality and salary. Then we introduce our dataset. We continue by refining our model to a specific form, and then presenting our results. Finally, we discuss the results and end with additional topics which can be explored.
